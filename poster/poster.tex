%==============================================================================
%== template for LATEX poster =================================================
%==============================================================================
%
%--A0 beamer slide-------------------------------------------------------------
\documentclass[final]{beamer}
\usepackage[orientation=portrait,size=a0,
            scale=1.25         % font scale factor
           ]{beamerposter}

\geometry{
  hmargin=2.5cm, % little modification of margins
}

%
\usepackage[utf8]{inputenc}

\linespread{1.15}
%
%==The poster style============================================================
\usetheme{sharelatex}

%==Title, date and authors of the poster=======================================
\title
[Final Presentation for ML Course, June 2018, Beijing]
{ % Poster title
AniML : Automatic scoring prediction \\
of Japanese Animes
}


\author{
Valentin KAO and Emile JONAS
}
\institute{Department of Computer Sciences\\
Tsinghua University\\
}
\date{\today}




\begin{document}
\begin{frame}[t]
%==============================================================================
\begin{multicols}{3}
%==============================================================================
%==The poster content==========================================================
%==============================================================================

\section{Introduction}
The Japanese anime industry is a major worldwide success thanks to the interest of other Asian countries. It is also, of intense interest to both Japanese economists and the public, who became more and more international, because of its high profits and entertainment nature. In 2017, the industry pulled in a record \$17.7 billion last year, and boosted by the famous hit, Your Name, growing exports and revenues \cite{web1}. According to the annual report from The Association of Japanese Animations (AJA),  the part of this growth, over the past 4 years, is principally due to expansion in the Chinese market. \\
Being able to predict the future smash hits is of great importance in decision making specially in the marketing. In spite of the fact that there are many factors that influence the success of an anime. It is not generally clear how they influence, this project endeavors to determine these factors using machine learning strategies by predicting a score out of 10 based on the data provided by the famous website MyAnimeList. \\

% \structure{Your text with scientific results or something...} $\hat H \Psi = E \Psi$

%\begin{equation}
%H = \sum_{i=1}^{N} h_{D}(i) + \sum_{j>i=1}^{N} C_{ij}
%\end{equation}

%On webpage~\cite{web}
%In Ref.~\cite{ref1}...
%In Refs.~\cite{ref1,ref2}...

\section{Data and correlation analysis}
There was no pre-existing complete dataset containing the animes, the anime studios, the voice actors associated as we wanted.
Indeed we decided to collect data from MyAnimeList website within more than 35,000 animes and more than 80,000 users preferences.
The mechanism to retrieve their data we found is called Jikan API, a third party MyAnimeList API that contain most information we
need The author of the project informs that his API has a 5000 limit per IP requests and that there was a regulation for building datasets:
a 4 to 5 seconds delay between requests in order to not impact other users.

Using Python and the third-party API, we build our dataset. The data collected was structured in such a way that most of its attributes and
information is organized and stored separately in different comma-separated values files. For example, all the animes were stored in the file
\textit{animes.csv} which contains information such as the title, the score, the genres. Moreover, we also collect the licensors, the voices
actors and the studios.  In this way, some kind of cleaning, intergration and preprocessing were required in order to to make great utilization
of the data for prediction through supervised machine learning techniques.

\\ data correlation plot
\\ correlation analysys
\\ normalize and standization
\\ selection of features

\section{Result and discussions}
Your text with scientific results or something...

%\vskip1ex
%\begin{table}
%\centering
%\caption{This is a table with scientific results.}
%\begin{tabular}{ccccc}
%\hline\hline
%1 & 2 & 3 & 4 & 5\\
%\hline
%aaa & bbb & ccc & ddd & eee\\
%aaaa & bbbb & cccc & dddd & eeee\\
%aaaaa & bbbbb & ccccc & ddddd & eeeee\\
%aaaaaa & bbbbbb & cccccc & dddddd & eeeeee\\
%1.000 & 2.000 & 3.000 & 4.000 & 5.000\\
%\hline\hline
%\end{tabular}
%\end{table}
%\vskip2ex


\subsection{SubSection}

Your text with scientific results or something...

%\vskip1ex
%\begin{figure}
%\centering
%\includegraphics[width=0.99\columnwidth]{logo.png}
%\caption{This is a picture with scientific results.}
%\end{figure}
%\vskip2ex

\subsection{SubSection, a very very very very very very long title}
Your text with scientific results or something...

\section{Summary and conclusions}
Your text with scientific results or something...


%==============================================================================
%==End of content==============================================================
%==============================================================================

%--References------------------------------------------------------------------

\subsection{References}

\begin{thebibliography}{99}

\bibitem{web1} \textit{Rafael Antonio Pineda (2017, October). "Anime Industry Takes in Record 2.0 Trillion Yen in 2016" retrieved 4th june of 2018 from} \url{https://www.animenewsnetwork.com/news/2017-10-25/anime-industry-takes-in-record-2.0-trillion-yen-in-2016/.123164}

\bibitem{ref1} J.~Doe, Article name, \textit{Phys. Rev. Lett.}

\bibitem{ref2} J.~Doe, J. Smith, Other article name, \textit{Phys. Rev. Lett.}


\end{thebibliography}
%--End of references-----------------------------------------------------------

\end{multicols}

%==============================================================================
\end{frame}
\end{document}
